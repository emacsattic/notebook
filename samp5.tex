% This is a sample Matlab Notebook.  $Id$
% It has some tex commands.
	
\documentclass{notebook}
\begin{document}

Type C-c C-f to run convert this to {\tt samp5.tex}, and then run latex on it. 


Here is a matrix.
\begin{verbatim}>>   a = [1 2 3; 2 1 5; -4 -1 3]  
a =
     1     2     3
     2     1     5
    -4    -1     3
\end{verbatim} It's inverse is defined here:
\begin{verbatim}>>  inv(a)  
ans =
   -0.2105    0.2368   -0.1842
    0.6842   -0.3947   -0.0263
   -0.0526    0.1842    0.0789
\end{verbatim}
Next, we will try some plotting.  Here is $\sin(t)$ for $0 < t < pi$.
\begin{verbatim}>>  t = [0:.01:pi];
plot(t, sin(t))\end{verbatim}
This is an example of a function in matlab.
\begin{verbatim}>>  function y = junk(x)
y = 3 * x .^ 2 - x ;(function junk.m saved)\end{verbatim}
$x=1^4$
\begin{verbatim}>>  plot(t, junk(t)) 
>>  eig(a) 
ans =
  -1.3352          
   3.1676 + 4.2925i
   3.1676 - 4.2925i
\end{verbatim}
Here is some concatinated:
\begin{verbatim}>> a = 1  
a =
     1

>> b = 2  
b =
     2
\end{verbatim}

Here is some short
{ \tt  ehll = 123.2314234   \, (
(no output yet))} 

 or
${ \tt z_2 = 4.55^3;  } $ or 
${ \tt  z_4 + z_2   \, (
ans =
   97.3364
)}  $

Here is:
{ \tt  [4  3]   \, (
ans =
     4     3
)} 
Or { \tt  a = [4  3]   \, (
ans =
     4     3
)} 

Within math delimeters: $ { \tt  z = 4+3   \, (
z =
     7
)} $
and without math delimeters { \tt  z = 4+3   \, (
z =
     7
)} (end)


With dollar signs ${ \tt [4 3 sin(3)]   \, (
ans =
    4.0000    3.0000    0.1411
)} $



\end{document}
